%投稿用クラス
\documentclass[uplatex,draft]{ipsjpapers}
%研究報告用クラス
%\documentclass[uplatex,techrep]{ipsjpapers}
%製版用クラス
%\documentclass{ipsjpapers}
%
\usepackage{graphicx}
\usepackage{graphics}
\usepackage{color}
%\usepackage{fancybox}
%\usepackage{ulem}
\usepackage{amsmath}
\usepackage{amssymb}
\usepackage{bm}
\usepackage{float}
%
%
% ユーザが定義したマクロなど.
\makeatletter
\let\@ARRAY\@array \def\@array{\def\<{\inhibitglue}\@ARRAY}
\def\<{\(\langle\)\nobreak}
\def\>{\nobreak\(\rangle\)}
\def\|{\verb|}
\def\Underline{\setbox0\hbox\bgroup\let\\\endUnderline}
\def\endUnderline{\vphantom{y}\egroup\smash{\underline{\box0}}\\}
\def\LATEX{\iLATEX\Large}
\def\LATEx{\iLATEX\normalsize}
\def\LATex{\iLATEX\small}
\def\iLATEX#1{L\kern-.36em\raise.3ex\hbox{#1\bf A}\kern-.15em
    T\kern-.1667em\lower.7ex\hbox{E}\kern-.125emX}
\def\LATEXe{\ifx\LaTeXe\undefined \LaTeX 2e\else\LaTeXe\fi}
\def\LATExe{\ifx\LaTeXe\undefined \iLATEX\scriptsize 2e\else\LaTeXe\fi}
\def\Quote{\list{}{}\item[]}
\let\endQuote\endlist
\def\TT{\if@LaTeX@e\tt\fi}
\def\CS#1{\if@LaTeX@e\tt\expandafter\string\csname#1\endcsname\else
	$\backslash$#1\fi}
%
\newcommand{\strref}[1]{文\nobreak{}(\ref{#1})}
\newcommand{\eqnref}[1]{式(\nobreak\ref{#1})}
\newcommand{\secref}[1]{\nobreak\ref{#1}節}
\newcommand{\chapref}[1]{\nobreak\ref{#1}章}
%
%
%数式の上下のスペース指定
%長い式用
\setlength\abovedisplayskip{12pt plus 3pt minus 7pt}
\setlength\belowdisplayskip{6.5pt plus 3.5pt minus 3pt}
%短い式用
\setlength\abovedisplayshortskip{0pt plus 3pt}
\setlength\belowdisplayshortskip{\abovedisplayskip}
%
%
%
%\renewcommand{\bib}{参考文献}
%
%
%
%\checklines	% 行送りを確認する時に使用
%
%
%
\begin{document}%{
%
% 和文表題
\title[{\protect\LaTeX} 情報処理学会による卒論ドラフト作成のガイド]%
  {阿南高専情報コース卒論ドラフトのサンプル}
%
% 英文表題
\etitle{A Sample of  Thesis Draft for the Course of Information Engineering at the Anan National College of Technology}

%
% 和文概要
\begin{abstract}
ここには,日本語のアブストラクトを書きます.
どういう背景のもとで,何のために,何をして,何を得られたのかをコンパクトに述べましょう.
\par
例.
近年,~~~の重要性が高まっている\cite{Gazoudenshi1}.
しかし,~~~という問題がある\multicite{Ohtake1}{Sagawa1}.
そこで本研究では,~~~について取り組む.
~~~を用いた実験により,~~~であることを示す.
\end{abstract}
%
%
% 英文概要
\begin{eabstract}
In this space, please write the abstract of your research in English.
\par
Example: In recent years, xxx is becoming more and more important\cite{Gazoudenshi1}.
However, there are some problems such as xxx\multicite{Ohtake1}{Sagawa1}.
In this paper, we addresses xxx.
Evaluation experiments using xxx shows that xxx.
\end{eabstract}
%
%
% 表題などの出力
\maketitle
%
%
%
%}{
%
%
%
% 本文はここから始まる
\section{はじめに}
\label{sec:intro}
%
\subsection{本研究の背景}
研究の技術的・社会的背景について,関連研究を適宜引用しながら述べましょう.
具体的には,自分の選んだ研究分野において,これまでどのような研究が行われてきたのか,一方で,何が未解決の問題なのか,などを述べましょう.
また,自分の研究テーマについて,なぜそのテーマを選んだのか,なぜそのテーマが重要なのか,などを述べましょう.
\par
%
\subsection{本研究の目的}
前述の背景を踏まえて,研究の目的についても述べましょう.
具体的には,何を問題にして,何を明らかにしようとしているのか,また,この研究を通じてどのような科学的・工学的な価値があるのか,などを述べましょう.
%
\subsection{本論文の構成}
最後に,論文の構成を示しましょう.
\par
例.本論文の構成を以下に示す.
\label{sec:fundamental}では,~~~の基本原理について述べる.
\label{sec:proposed}では,本研究で提案する~~~について述べる.
\label{sec:experiment}では,提案手法を~~~を用いた実験によって評価する.
\label{sec:discussion}では,実験結果について考察する.
最後に,\label{sec:conclusion}では,本研究の結論を述べるとともに,今後の課題について検討する.
%
%
\section{基本技術と原理}
\label{sec:fundamental}
研究で用いる技術や理論の基本的な事項・原理などについて述べましょう.
数式や図を適宜用いるとともに,重要な論文があれば引用しましょう.
%
\subsection{章立ての注意点その1}
これ以降の各章はこのように幾つかの節に分け,分かりやすく構成しましょう.
まったく節のない章は,一般に有り得ません.
%
\subsection{章立ての注意点その2}
同様に,1つしか節のない章も,一般に有り得ません.
節は2つ以上に分けましょう.
%
\section{提案手法}
\label{sec:proposed}
この章では,提案する手法やシステムについて述べます.
数式や図を用いて,提案手法の原理や,提案するシステムの構成について説明しましょう.
ソースコード等がある場合には,付録に示しましょう.
ただし,コードが膨大な場合など,紙面に掲載することが不適切な場合には,付録に載せる代わりに,電子データをZIPにまとめて卒研ポートフォリオにアップロードしましょう.
%
%
\section{評価実験}
\label{sec:experiment}
この章では,実験について述べます.
%
\subsection{実験条件}
実験条件(使用した機器,ソフトウェア,データ,試料)について述べましょう.
機器やソフトウェアについては,バージョン情報などを示しましょう.
データや試料については,データの諸元(どのようなデータを,どのような環境で,どれくらい用いたのか,その他統計量など)を示しましょう.
後でドラフトを読んだ人が実験を再現できるように,実験手順を詳細に書きましょう.
また,前述のソースコードと併せて,実験を再現するために必要なデータがあれば,ZIPでまとめて卒研ポートフォリオにアップロードしましょう.
ただし,moodleの容量制限を越えるような膨大なデータは含める必要はありません.
%
\subsection{実験結果}
実験結果について述べましょう.
プログラムの実行結果や実験から得られたデータを,図表を用いて分かりやすく示しましょう.
その際は,「~~~の実験結果を図1に示す」というように,必ず図番号・表番号を明記しましょう.
%
\section{考察}
\label{sec:discussion}
実験結果から分かることについて説明しましょう.
考察は論文の contribution を示す最も重要な部分です.
予め立てた仮説に対して実際の実験結果はどうだったのか,実験結果から科学的・工学的にどのような価値が得られるか,などを明確に述べましょう.
ドラフトとしては,うまく行かなかった部分も重要な情報ですので,漏れなく書いて下さい.
%
%
\section{結論}
\label{sec:conclusion}
\subsection{本研究のまとめ}
研究のまとめについて述べましょう.
%
\subsection{今後の課題}
今後の課題(システムに追加すべき機能,未解決の問題点,提案手法の発展の可能性など)について述べましょう.
%
%
\begin{thebibliography} {30}
%
%
\bibitem{Gazoudenshi1}
西田 友是,近藤 邦雄,藤代 一成:
ビジュアルコンピューティング,
画像電子学会,東京電機大学出版局(2006).
%
\bibitem{Ohtake1}
Ohtake,Y.,Belyaev,A.G.,Alexa,M.,Turk,G and Seidel,H.-P.:
Multi-level partition of unity implicits,
ACM Transactions on Graphics(Porc. SIGGRAPH 2003),
Vo.22, No.3 pp.\ 463--470(1999).
%
\bibitem{Satoh1}
佐藤 善隆,羽石 秀昭:レベルセット法を用いた医用画像セグメンテーション,
電子情報通信学会技術研究報告,MI2004-81(2005-1),pp.\ 1--6(2005).
%
\bibitem{Sagawa1}
佐川 立昌, 池内 克史:符号付距離場の整合化による形状モデル補間手法,
電子情報通信学会論文誌 D-II,Vol.~J88-D-II, No.~3,pp.\ 541--551(2005).
%
\bibitem{Evans1}
Evans,L.C. and Spruck,J.:
Motion of level sets by Mean Curvature I,
J.Diff.Geom., Vo.33, No.635(1991).
%
%
%
\end{thebibliography}
%
%
%
%}{
%
%
%
%付録の始まり
\appendix
\section{プログラムソース}
\begin{itemize}
\item 提案システムのソースコードなど,重要なものがあればここに掲載しましょう.
\item ただし,コードが膨大な場合など,紙面に掲載するのが不適切なものは,電子データをZIPにまとめて卒研ポートフォリオにアップロードしましょう.
\end{itemize}
%
\section{設計図}
\begin{itemize}
\item 機器の設計図などがあれば,ここに掲載しましょう.
\item プログラムのフローチャートやUML図などを載せる場合もあります.
\end{itemize}
%
\section{使用したデータ}
\begin{itemize}
\item 重要な実験データなどがあれば,ここに掲載しましょう.
\item ただし,ソースコードと同様に,紙面に掲載するのが不適切なものは,電子データをZIPにまとめて卒研ポートフォリオにアップロードしましょう.
\end{itemize}
%
\end{document}
